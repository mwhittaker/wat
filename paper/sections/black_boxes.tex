\section{Black Boxes}\seclabel{BlackBoxes}
Now that we've defined \watprovenance{}, we turn to the matter of computing it.
Real-world distributed systems are typically a composition of two types of
components: (1) application specific services that are written from scratch
(i.e.\ ``business logic''), dubbed \defword{white boxes}, and (2) widely used
open source services that are not written by the application developer, dubbed
\defword{black boxes}, (e.g., Redis, Postgres). In this section, we discuss how
to compute the \watprovenance{} of black boxes. In \secref{WhiteBoxes}, we
discuss white box \watprovenance{}.

\subsection{Provenance Specifications}
Automatically computing the \watprovenance{} for an \emph{arbitrary} black box
is more or less hopeless. In order to do so, we would have to analyze an
arbitrary piece of code written in an arbitrary programming language and infer
an abstract state machine that accurately models the black box. Then, to
compute the \watprovenance{} of a particular request with respect to a given
trace, we would have to perform some sort of complex code analysis over a large
body of code. Worse, we may not even have access to the source code of the
black box!  For example, we don't have access to the source code of cloud
services like Amazon S3 or Google Cloud Spanner.

% given an arbitrary black box written in an arbitrary language, it is a near impossibility to efficiently extract the wat provenance in an efficient way. In order to do so, we would have take in an arbitrary piece of code and automatically infer a corresponding state machine. Then, given a trace, we would then have to perform a complex code analysis over a large body of code to infer the wat provenance. For example, Redis is over 50,000 lines of C code that we would have to analyze. Worse yet, we may not even have the source code! For example, if we use a service like S3 in the cloud, we don't even have access to the code to profile.
\newcommand{\kvget}{\texttt{get}}
\newcommand{\kvset}{\texttt{set}}

Though automatically computing the \watprovenance{} for an \emph{arbitrary}
black is intractable, we can take advantage of the fact that most real-world
black boxes are far from arbitrary. Many black boxes have complex
implementations but are designed with very simple interfaces.  This allows us
to sidestep the issue of \emph{inferring} \watprovenance{} from an
implementation and instead \emph{specify} \watprovenance{} directly from an
interface. That is, we can write a \defword{\watprovenance{} specification}: a
function that---given a trace $T$ and request $i$---directly returns the
\watprovenance{} $\Wat{M, T, i}$ for a black box modeled as state machine $M$.

For example, if we restrict our attention to the \kvget{} and \kvset{} API of
Redis (as with \exampleref{WatExampleXyx}), then the \watprovenance{}
specification of a \kvget{} request is trivial: \textbf{the \watprovenance{} of
a \kvget{} request for key $k$ includes only the most recent \kvset{} to
$k$}. Redis is implemented in over 50,000 lines of C, so automatically
inferring the \watprovenance{} of a \kvget{} request is intractable. This
\watprovenance{} specification avoids the intractability and instead specifies
the \watprovenance{} in a single line of text.

Moreover, codifying this one line \watprovenance{} specification is expectedly
straightforward. The programming language in which we write \watprovenance{}
specifications is unimportant. In \figref{RedisProvSpec}, we provide a simple
implementation of the one line \watprovenance{} specification in Python. The
specification, \texttt{get\_prov}, takes in a trace \texttt{T} and a \kvget{}
request \texttt{i} for key $k$. Redis requests are represented as objects of
type \texttt{Request} with subclasses \texttt{GetRequest} and
\texttt{SetRequest}.  \texttt{get\_prov} iterates through the trace in reverse
order, looking for a \kvset{} request to key $k$. If such a \kvset{} request is
found, \texttt{get\_prov} returns it. Otherwise, \texttt{get\_prov} returns an
empty witness.
% Recall that the \watprovenance{} of a request is a set of witnesses closed
% under supertrace in $T$, which is why \texttt{get\_prov} returns either a set
% of the singleton witness \texttt{[a]} or a set of the empty witness
% \texttt{[]}.

{\begin{figure}[ht]
  \centering
  \inputpython{figures/redis_prov_spec.py}
  \caption{Redis \kvget{} request \watprovenance{} specification}
  \figlabel{RedisProvSpec}
\end{figure}
}

In \secref{DebuggingHeterogeneousSystems}, we present a prototype
implementation of a system for writing \watprovenance{} specifications and
describe the details of concretely how traces are collected and how
\watprovenance{} specifications are written. In this section, we omit the
details and focus on the concepts behind \watprovenance{} specifications.

\subsection{Examples}
The \watprovenance{} specification of a Redis \kvget{} request is particularly
simple, but this simplicity is the rule, not the exception. We now survey a
variety of popular black boxes and demonstrate that their \watprovenance{}
specifications are mostly very simple.

\paragraph{Stateless Services}
A \defword{stateless service} is a service that can be modelled as a state
machine with a single state. Given a request, a stateless service always
produces the same reply, no matter what other requests it has already serviced.
For example, a web server serving a static website is stateless; it replies to
all requests with the same website. Similarly, cloud services like Google Cloud
Vision API and Google Cloud Speech-to-Text are also stateless. Given an image
or audio clip, these services deterministically return a description of the
image or a transcription of the audio. \Watprovenance{} specifications of a
stateless service are trivial. Requests are completely independent, so the
\watprovenance{} of any request consists only of the empty witness.

\paragraph{Key-Value Stores}
We've already seen a \watprovenance{} specification for the \kvget{} and
\kvset{} API of Redis. This specification works equally well for any other
key-value store like Voldemort, Riak, or Memached. Moreover, we can easily
extend our \watprovenance{} specification to handle more of Redis' API. For
example, consider the operations \texttt{append}, \texttt{decr},
\texttt{decrby}, \texttt{incr}, and \texttt{incrby} which all modify the value
associated with a particular key. With these operations present in a trace, the
\watprovenance{} specification for a \kvget{} request to key $k$ now includes
the most recent set to $k$ and all subsequent modifying operations to key $k$.

\paragraph{Object Stores}
We can also write \watprovenance{} specifications for storage systems that are
more complex than key-value stores. For example, consider an object store like
Amazon S3 where users can create, move, copy, list, and get buckets and
objects. The \watprovenance{} specification for the get of an object $o$ in
bucket $b$ includes the most recent creation of the bucket $b$ (either by
creation or moving) and the most recent creation of $o$ (again, either by
creation or moving). The \watprovenance{} of a request to list the contents of
a bucket includes the most recent creation of the bucket, the most recent
creation of every object in the bucket, and the deletion of any object that was
previously in the bucket.

\paragraph{Distributed File Systems}
We can also specify the \watprovenance{} of a distributed file system like NFS
or HDFS. A \watprovenance{} specification of a request to read a byte range
from a file includes the most recent creation of the file, the most recent
creation of the parent directories of the file, and the most recent writes that
overlap with the requested byte range.

\paragraph{Coordination Services}
Systems use coordination services like Apache
Zookeeper~\cite{hunt2010zookeeper} and Chubby~\cite{burrows2006chubby} for
leader election, mutual exclusion, etc. Take Zookeeper as an example.
Zookeeper's API resembles that a of a file system; users can create, delete,
write, and read file-like objects called znodes. Though the implementation of
Zookeeper and HDFS are radically different, their APIs (and thus their
\watprovenance{} specifications) are similar. For example, the \watprovenance{}
specification of a request to read a znode includes the most recent creation of
the znode and the most recent creations of all ancestor znodes.

\paragraph{Load Balancers}
Consider a load balancer, like HAProxy, that is balancing load across a set
$s_1, \ldots, s_n$ of $n$ servers. Periodically, a server $s_i$ sends a
heartbeat to the load balancer that includes $s_i$'s average load for the last
five minutes. Whenever the load balancer receives a message from a client, it
forwards the message to the server $s_i$ that is least loaded. Modelling the
forwarding decision $s_i$ as the output of the load balancer, the
\watprovenance{} specification for the forwarding decision includes the most
recent heartbeat message from the least loaded server.

\subsection{Discussion}
from s3 to ..., we see that most apis are very simple and often overlapping. For example, zookeeper by being a file system lets us leverage ideas from previous specs. This is not just a coincidence though. Systems design their APIs to be simple and familiar, meaning that many are simple to specify and many overlap.

Moreover, only one person has to write the black box spec for a particular black box, not every dev. Thus, if someone wrote a redis spec, everyone could use it. Moreover, the spec is invariant to the implementation of Redis, we do not have to change the spec every time the implementation changes. in secref eval, we see that some approaches do require that. Also, the overhead of writing these specs is significantly, significantly lower than the overhead of other existing methods, as well see in the eval section

Even though specs are simple, it is still possible for a user to incorrectly write a spec. Fortunately, we can write simple unit tests for wat prov specs. The main idea is that given a simple implementation of a state machine and a trace, we can manually compute the wat provenance. Then, we can check this wat provenance against the wat provenance returned by the specification. just like unit tests, this allows to sometimes conclude that the wat provenance specification is incorrect, though it never allows to say the spec is definitely correct.

An astute reader might note that this contradicts what we said earlier. We're saying we can automatically extract the wat provenance. This approach is useful for testing but not practical for getting the wat prov of a real system. The runtime of the algorithm is superexponential, meaning that for any realistically sized traces, this approach would simply not work. Still, it is useful for small unit tests.

\subsection{Limitations}
buggy black boxes:
  - we have so far tacitly assumed that a black box faithfully implements its api. If a black box is buggy, then its actual wat provenance may not be the same as what we've specified. Still, bugs in large open source projects are rare and are often fixes after being found.

nondeterministic state machines:
  - wat provenance assumes a deterministic state machine, but some state machines are nondeterministic. for example, consider a load balancer which randomly forwards traffic to servers. modelling such a thing as a deterministic state machine is impossible, so it doesn't fit our model of wat provenance very well. we think an interesting aveneue for future work will be to expand our notions of wat prov and wat prov specs to nondeterminsitic state machines

nontrivial specifications:
  - as we've noted, specifying the provenance for an arbitray black box is hard. As a contrived example, consider a server which takes in a series of programs as inputs and later returns whether one of the programs terminates. Writing a specification is tantamount to determining which of these programs terminate which is likely difficult, and undecidable in general. fortunately, many open source black boxes have simple apis, but some might be hard to specify.
