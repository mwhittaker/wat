\newcommand{\dedalusplus}{Dedalus$^+$}

\section{White Boxes}\seclabel{WhiteBoxes}
Recall that real-world distributed systems are composed of (a) widely used
black boxes that are developed and maintained by a handful of developers and
(b) application-specific white boxes that are written from scratch. In
\secref{BlackBoxes}, we saw that automatically inferring the \watprovenance{}
of a black box is intractable. Is the same true for white boxes? Almost, but
not quite. If we implement a white box from scratch, we have the opportunity to
implement it in a programming language for which we \emph{can} automatically
extract \watprovenance{}.

In this section, we propose one such programming language. However, we also see
that this programming language is not very expressive and that making the
language more expressive forfeits our ability to easily extract
\watprovenance{}. The expressiveness of a language is fundamentally at odds
with our ability to extract the \watprovenance{} of a program written in it.

\subsection{\dedalusplus}
Dedalus is a variant of Datalog that introduces notions of time and mutable
state~\cite{alvaro2011dedalus, marczak2012confluence, alvaro2013declarative}.
Bloom is a concrete implementation of Dedalus that has been used to program a
variety of distributed systems, including key-value stores, state machine
replication protocols, and distributed shopping
carts~\cite{alvaro2011consistency, conway2012logic}. In this section, we
describe how we can automatically extract the \watprovenance{} of a program
written in \defword{\dedalusplus}: the variant of Dedalus without negation.

{\begin{figure}[t]
  \centering
  {
    \small
    \begin{Python}[gobble=3]
      Users(handle: str, name: str)
      Following(follower: str, followee: str)

      def signup(handle: str, name: str):
        Users += {(handle, name)}
        return {("ok")}

      def follow(me: str, them: str):
        Following += {(me, them)}
        return {("ok")}

      def following(me: str):
        return SELECT U.name
               FROM Following F, Users U
               WHERE F.follower=me AND U.handle=followee
    \end{Python}
  }
  \caption{\dedalusplus{} Twitter clone}
  \figlabel{DedalusTwitter}
\end{figure}
}

A \dedalusplus{} state machine stores its state in a collection of typed
relations, much like a relational database. Upon receiving a request, a
\dedalusplus{} state machine executes a sequence of relational expressions to
(a) update its state and (b) compute a response for the request. For example,
consider the simple Twitter clone shown in \figref{DedalusTwitter} that is
written with a Python-like syntax of \dedalusplus{}. The program stores its
state in two relations: (1) \texttt{Users} which records the Twitter handle and
name of all users and (2) \texttt{Following} which records the handles that a
particular user is following. The Twitter clone supports three types of
requests:
\begin{itemize}
  \item
    A \texttt{signup(handle, name)} request creates an account with handle
    \texttt{handle} for a user named \texttt{name}.  Concretely,
    \texttt{signup} adds the tuple \texttt{(handle, name)} to the
    \texttt{Users} relation and returns an acknowledgement tuple
    \texttt{("ok")}. For example, \texttt{signup("dbguy", "Edgar Codd")}
    registers a user named Edgar Codd with handle \texttt{dbguy}.
  \item
    A \texttt{follow(me, them)} request requests that the user with handle
    \texttt{me} follows the user with handle \texttt{them}. It adds the tuple
    \texttt{(me, them)} to the \texttt{Following} relation and returns an
    acknowledgement tuple \texttt{("ok")}.
  \item
    A \texttt{following(me)} request executes a SQL query to return the names
    of every user that is being followed by the user with handle \texttt{me}.
\end{itemize}

For any execution of a \dedalusplus{} program, like the Twitter clone in
\figref{DedalusTwitter}, we can generate an accompanying \defword{provenance
graph}~\cite{alvaro2015lineage}. A provenance graph records the data
dependencies for every tuple that is created during the execution of the
program.

{\begin{figure*}[ht]
  \centering
  \begin{subfigure}[c]{0.3\textwidth}
    \small
    \centering
    \begin{tabular}{rcl}
      $T$  & $=$ & $r_1$: \texttt{signup(dbguy, Codd)}; \\
           &     & $r_2$: \texttt{signup(baller, Jordan)}; \\
           &     & $r_3$: \texttt{signup(mldude, Jordan)}; \\
           &     & $r_4$: \texttt{follow(dbguy, baller)}; \\
           &     & $r_5$: \texttt{follow(dbguy, mldude)}; \\
           &     & $r_6$: \texttt{following(dbguy)} \\
    \end{tabular}
    \caption{Trace}
    \figlabel{TwitterCloneTrace}
  \end{subfigure}%
  %
  \begin{subfigure}[c]{0.2\textwidth}
    \centering
    \small

    \texttt{Users} \\
    \begin{tabular}{l|l|l|}
      \cline{2-3}
      & \cellcolor{flatgray!30} \textbf{\texttt{handle}}
      & \cellcolor{flatgray!30} \textbf{\texttt{name}} \\\cline{2-3}
      $t_1$: & \texttt{baller} & Jordan \\
      $t_2$: & \texttt{mldude} & Jordan \\
      $t_3$: & \texttt{dbguy}  & Codd \\\cline{2-3}
    \end{tabular}

    \vspace{3mm}

    \texttt{Following} \\
    \begin{tabular}{l|l|l|}
      \cline{2-3}
      & \cellcolor{flatgray!30} \textbf{\texttt{follower}}
      & \cellcolor{flatgray!30} \textbf{\texttt{followee}} \\\cline{2-3}
      $t_4$: & \texttt{dbguy}    & \texttt{baller} \\
      $t_5$: & \texttt{dbguy}    & \texttt{mldude} \\\cline{2-3}
    \end{tabular}
    \caption{Relations}
    \figlabel{TwitterCloneRelations}
  \end{subfigure}%
  %
  \begin{subfigure}[c]{0.5\textwidth}
    \small
    \centering

    \tikzstyle{t}=[minimum width=2.3cm, draw, align=center, rectangle split,
                   rectangle split parts=2, inner sep=1pt]
    \tikzstyle{d}=[draw]
    \tikzstyle{provedge}=[-latex]
    \tikzstyle{label}=[inner sep=0, circle, fill=white, fill=flatgray!50]
    \newcommand{\twoparts}[2]{\texttt{#1}\nodepart{second}\texttt{#2}}

    \begin{tikzpicture}[yscale=0.8, xscale=2.75]
      % Nodes.
      \node[t] (a1) at (0, 1) {\twoparts{follow}{(dbguy, baller)}};
      \node[t] (a2) at ($(a1) + (1, 0)$) {\twoparts{Following}{(dbguy, baller)}};
      \node[t] (b1) at ($(a1) - (0, 1)$) {\twoparts{signup}{(baller, Jordan)}};
      \node[t] (b2) at ($(b1) + (1, 0)$) {\twoparts{Users}{(baller, Jordan)}};
      \node[t] (d1) at ($(b1) - (0, 1)$) {\twoparts{signup}{(mldude, Jordan)}};
      \node[t] (d2) at ($(d1) + (1, 0)$) {\twoparts{Users}{(mldude, Jordan)}};
      \node[t] (e1) at ($(d1) - (0, 1)$) {\twoparts{follow}{(dbguy, mldude)}};
      \node[t] (e2) at ($(e1) + (1, 0)$) {\twoparts{Following}{(dbguy, mldude)}};
      \node[d] (f1) at ($(b2) + (0.75, 0.5)$) {};
      \node[d] (f2) at ($(e2) + (0.75, 0.5)$) {};
      \node[draw] (out) at ($(d2) + (1.25, 0.5)$) {\texttt{(Jordan)}};

      % Labels.
      \node[label] (t1label) at (b2.north west) {$t_1$};
      \node[label] (t2label) at (d2.north west) {$t_2$};
      \node[label] (t4label) at (a2.north west) {$t_4$};
      \node[label] (t5label) at (e2.north west) {$t_5$};
      \node[label] (r2label) at (b1.north west) {$r_2$};
      \node[label] (r3label) at (d1.north west) {$r_3$};
      \node[label] (r4label) at (a1.north west) {$r_4$};
      \node[label] (r5label) at (e1.north west) {$r_5$};

      % Edges.
      \draw[provedge] (out) edge (f1) (f1) edge (a2.east) (a2) edge (a1);
      \draw[provedge]                 (f1) edge (b2.east) (b2) edge (b1);
      \draw[provedge] (out) edge (f2) (f2) edge (d2.east) (d2) edge (d1);
      \draw[provedge]                 (f2) edge (e2.east) (e2) edge (e1);
    \end{tikzpicture}

    \caption{Provenance graph}
    \figlabel{TwitterCloneProvGraph}
  \end{subfigure}
  \caption{%
    An example execution of the Twitter clone from \figref{DedalusTwitter}
  }
  \figlabel{TwitterCloneExample}
\end{figure*}
}

For example, consider the trace, relations, and provenance graph illustrated in
\figref{TwitterCloneExample}. The trace $T$ in \figref{TwitterCloneTrace}
consists of six requests: $r_1$ through $r_6$. Requests $r_1$ through $r_3$
register three users: Edgar Codd (\texttt{@dbguy}), Michael Jordan
(\texttt{@baller}), and Michael Jordan (\texttt{@mldude}). Requests $r_4$ and
$r_5$ are requests from Edgar Codd to follow both the Michael Jordans.  The
contents of the \texttt{Users} and \texttt{Following} relations after requests
$r_1$ through $r_5$ have been executed are shown in
\figref{TwitterCloneRelations}. Finally, the trace $T$ includes a request $r_6$
from Edgar Codd to list the names of everyone he is following. This request
returns a single tuple \texttt{(Jordan)}, and the corresponding provenance
graph for this tuple in shown in \figref{TwitterCloneProvGraph}.

The provenance graph indicates that the \texttt{(Jordan)} tuple is produced
twice: once from the join of tuples $t_4$ and $t_1$ and once from the join of
tuples $t_2$ and $t_5$. Moreover, the provenance graph indicates the request
that created each of these tuples. For example, the $r_4$ request created tuple
$t_4$, the $r_2$ request created tuple $t_1$, etc.

The \watprovenance{} of the \texttt{following(dbguy)} request is the set
$\set{T_{24}, T_{35}}$ of two witnesses $T_{24} = r_2r_4$ and $T_{35} =
r_3r_5$. These two witnesses are exactly the two witnesses that we get if we
walk backwards through the provenance graph in \figref{TwitterCloneProvGraph}
beginning at our output tuple \texttt{Jordan}. This is not a coincidence. Given
any output produced by a state machine written in \dedalusplus{}, we can always
walk backwards through the provenance graph rooted at the output and
efficiently extract the corresponding \watprovenance{}.

\subsection{Limitations and Alternatives}
TODO:
  - Admit that \dedalusplus{} is not the most expressive language in the world.
    We can't really delete anything.
  - Say that adding in negation to get the full power of Dedalus makes the
    language to complication to get wat provenance out of in a nice way.
    Negative provenance is not super well understood.
  - Say that there are alternatives to wat provenance that are well suited to
    more expressive languages like Dedalus. Mention network provenance as an
    example.
  - Say that even if you write a program in a normal programming language, it's
    probably still ok. You can use normal debugging tools on it to walk through
    what's going on. It's ad-hoc but if the program is simple, it's probably
    ok.  Contrast this with black boxes, where this is very onerous because you
    didn't write the code.

Really emphasize that we're not suggesting that people write in dedalus+ programs; its more a proof of concept showing the most expressive language that it can be done for
