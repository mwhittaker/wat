\section{\fluent{}}\seclabel{Fluent}
In this section, we turn theory to practice with \fluent{}: a prototype system
that implements the ideas presented in \secref{BlackBoxes} and
\secref{WhiteBoxes}. \fluent{} has two core components. First, \fluent{}
includes a Dedalus-like C++ domain specific language for implementing white
boxes for which we can automatically extract network provenance. Second,
\fluent{} includes a mechanism with which to write black box \watprovenance{}
specifications. We describe these two components and then return to our
motivating example from \secref{WatProvenanceExample} to see how \fluent{}
leverages \watprovenance{} to make debugging distributed systems significantly
easier.

\subsection{White Boxes}
Fluent includes a Dedalus-like C++ domain specific language for implementing
white box business logic and an accompanying browser-based debugger. Fluent
programs represent their state as a collection of typed relations and express
computation as a series of relational algebra expressions. Every tuple in a
\fluent{} program is tagged with the logical timestamp at which it was created
and deleted. This enables a developer to \emph{debug across time}. Using the
debugger, a developer can walk forwards and backwards through the execution of
a \fluent{} program inspecting the state of the program at every point in time.
%
For example, \figref{FluentHistory} illustrates the history of a relation
across three logical points in time. As the developer walks forwards from
logical time 10 to logical time 12, they can observe that the tuple
\texttt{(data, base)} was inserted at logical time 11 and that the tuple
\texttt{(foo, bar)} was deleted at logical time 12.

{\begin{figure}[ht]
  \centering
  \small

  \begin{minipage}{0.3\columnwidth}
    Logical Time 10

    \begin{tabular}{|l|l|}
      \hline
      \rowcolor{flatgray!30}
      \textbf{\texttt{key}} & \textbf{\texttt{value}} \\\hline
      \texttt{foo}          & \texttt{bar} \\
      \texttt{zardoz}       & \texttt{connery} \\
                            & \\\hline
    \end{tabular}
  \end{minipage}\hspace{0.3cm}
  \begin{minipage}{0.3\columnwidth}
    Logical Time 11

    \begin{tabular}{|l|l|}
      \hline
      \rowcolor{flatgray!30}
      \textbf{\texttt{key}} & \textbf{\texttt{value}} \\\hline
      \texttt{foo}          & \texttt{bar} \\
      \texttt{zardoz}       & \texttt{connery} \\
      \rowcolor{flatgreen!20}
      \texttt{data}         & \texttt{base} \\\hline
    \end{tabular}
  \end{minipage}\hspace{0.3cm}
  \begin{minipage}{0.3\columnwidth}
    Logical Time 12

    \begin{tabular}{|l|l|}
      \hline
      \rowcolor{flatgray!30}
      \textbf{\texttt{key}} & \textbf{\texttt{value}} \\\hline
      \rowcolor{flatred!20}
                            &              \\
      \texttt{zardoz}       & \texttt{connery} \\
      \texttt{data}         & \texttt{base} \\\hline
    \end{tabular}
  \end{minipage}

  \caption{Relational history}
  \figlabel{FluentHistory}
\end{figure}
}

When a developer notices a tuple $t$ of interest at a particular point in time,
they can click on the tuple and view the network provenance graph rooted at $t$
(like the one in \figref{TwitterCloneProvGraph}). Navigating the provenance
graph, they can view the tuples that contributed to the creation of tuple $t$.
These ancestor tuples in the provenance graph include other tuples that are
colocated with tuple $t$, but they also include tuples on other \fluent{} white
boxes in the distributed system. Thus, \fluent{} also enables a developer to
\emph{debug across space}, similar to distributed tracing tools
like Dapper~\cite{sigelman2010dapper} and X-Trace~\cite{fonseca2007x}.

\subsection{Black Boxes}
The provenance graph for a piece of data can be generated automatically if the
data is produced by a \fluent{} white box. However, if the data is produced by
a black box component, then \fluent{} relies on \watprovenance{} specifications
to generate the provenance of the data. To write a \watprovenance{}
specification for a black box, a developer first wraps the black box in a
\fluent{} shim, as illustrated in \figref{FluentShim}. To send a request to a
black box, a white box sends the request to the black box's shim. The shim
forwards the request to the black box and relays the corresponding response
back to the white box. The shim also persists the sequence of inputs that it
receives in a relational database, forming a trace. A developer then registers
a \watprovenance{} specification as a simple script that, given a particular
input, queries the relational database for the corresponding \watprovenance{}.
Currently, \fluent{} allows \watprovenance{} specifications to be written
directly in SQL or written in Python. For example, the \watprovenance{}
specification for Redis \texttt{get} requests is similar to the specification
in \figref{RedisProvSpec}, except that the trace is stored in a relational
database instead of materialized in a list.

{\begin{center}
  \tikzstyle{minw}=[minimum width=2in]
  \tikzstyle{minh}=[minimum height=2in]
  \tikzstyle{wb}=[align=center, draw=black, fill=white, text=black]
  \tikzstyle{bb}=[align=center, draw=black, fill=black, text=white]

  \begin{tikzpicture}[scale=4]
    \node[wb, minw, minh] (wb) at (0, 0) {Client};
    \node[wb, minw, rotate=90] (shim) at (2.5, 0) {Shim};
    \node[bb, minw, minh] (bb) at (4, 0) {Server};

    \tikzstyle{e}=[thick, -latex]
    \tikzstyle{n}=[inner sep=0, semithick]
    \draw[e]
      ($(wb.north east)!0.25!(wb.south east)$)
      to node[n, above=0.1cm] {1}
      ($(shim.north west)!0.75!(shim.north east)$);
    \draw[e]
      ($(shim.south west)!0.75!(shim.south east)$)
      to node[n, above=0.1cm] {2}
      ($(bb.north west)!0.25!(bb.south west)$);
    \draw[e]
      ($(bb.north west)!0.75!(bb.south west)$)
      to node[n, below=0.1cm] {3}
      ($(shim.south west)!0.25!(shim.south east)$);
    \draw[e]
      ($(shim.north west)!0.25!(shim.north east)$)
      to node[n, below=0.1cm] {4}
      ($(wb.north east)!0.75!(wb.south east)$);
  \end{tikzpicture}
\end{center}
}

\subsection{Our Motivating Example}
\newcommand{\systemname}{ZardozBook}
We now return to the motivating example from \secref{WatProvenanceExample} and
explore how \fluent{} enables us to diagnose why Bob was able to see Ava's mean
comment. We assume that Ava and Bob's \systemname{} clients, the load balancer,
and the application servers are \fluent{} white boxes. We also assume that we
have written \watprovenance{} specifications for the Redis and Postgres black
boxes.

\begin{itemize}
  \item
    We begin by inspecting the history of Bob's \systemname{} client. We find
    the point at time in which he received Ava's mean comment. We select the
    message and view it's provenance graph.
  \item
    The provenance graph---which is automatically generated from the white
    boxes---shows that Bob received the message from the load balancer which in
    turn forwarded the message from application server $s_3$ which in turn read
    the message from its Redis cache.
  \item
    The \fluent{} debugger invokes Redis' \watprovenance{} specification. The
    resulting \watprovenance{} indicates that the Redis instance returned the
    mean comment because it was previously inserted by application server
    $s_3$.
  \item
    Continuing along the provenance graph, we find that application server
    $s_3$ inserted the mean message into its Redis cache because it received
    the comment from the centralized Postgres database as part of a cache
    refresh.
  \item
    The \fluent{} debugger invokes the \watprovenance{} specification of the
    Postgres database and finds that it produced the mean comment because it
    was previously inserted by application server $s_2$.
  \item
    Following the provenance graph further back through time, we find $s_2$
    inserted the mean comment into the Postgres database because it previously
    received the comment from the load balancer which in turn received the
    comment from Ava's \systemname{} client.
  \item
    Inspecting the history of Ava's \systemname{} client, we see that Ava
    unfriended Bob before posting this mean comment. If we follow this unfriend
    request forwards through the provenance graph, we find it arrives at
    application server $s_1$ (unlike the comment, which was sent to server
    $s_2$). Then, jumping to the Postgres database's trace, we see that the
    mean comment arrives before the unfriend request.
\end{itemize}

From this analysis, we discover that because the load balancer can forward
requests from a single client to different application servers and because
application servers lazily synchronize with the centralized Postgres
repository, client requests can be effectively reordered. Using this insight, a
\systemname{} engineer can enable sticky sessions in the load balancer,
ensuring that the load balancer forwards requests within a single session to a
single application server. This fixes the bug and prevents Bob from
experiencing any future race condition woes.
